\documentclass[12pt]{article}
\usepackage{fullpage}

\newcommand{\WEB}{\textsf{WEB}}
\newcommand{\Caml}{\textsf{Caml}}
\newcommand{\ocamlweb}{\textsf{ocamlweb}}

\begin{document}

%%% titre %%%%%%%%%%%%%%%%%%%%%%%%%%%%%%%%%%%%%%%%%%%%%%%%%%
\title{ocamlweb: a literate programming tool \\ 
               for Objective Caml}
\author{Jean-Christophe Filli\^{a}tre \\
        \normalsize\texttt{www.lri.fr/\~{}filliatr}}
\date{}
\maketitle
%%%%%%%%%%%%%%%%%%%%%%%%%%%%%%%%%%%%%%%%%%%%%%%%%%%%%%%%%%%%

\section{Introduction}

%%%%%%%%%%%%%%%%%%%%%%%%%%%%%%%%%%%%%%%%%%%%%%%%%%%%%%%%%%%%

\section{Principles}

Documentation is inserted into \Caml\ files as \emph{comments}.  Thus
your files may compile as usual, whether you use \ocamlweb\ or not.
\ocamlweb\ presupposes that the given \Caml\ files are well-formed (at
least lexically, but also syntactically if you want the cross
references to be correct).  \ocamlweb\ understands each \Caml\ file as
a sequence of paragraphs, each paragraph being either a piece code or a
piece of documentation.  Documentation starts and ends with \Caml\ 
comment's delimiters \texttt{(*} and \texttt{*)}, and is a regular
\LaTeX\ text. Code starts with any characters sequence other than
\texttt{(*} and ends with an empty line.

Documentation and code are displayed in \emph{sections} --- not in the
usual sense of sections in \LaTeX\ --- numbered
consecutively starting with~1. A new section is started at the
beginning of each file. You can also start a new section manually (see
the paragraph below about controls). The role of sections is to be
used as references in the various indexes produced at the end of the
document.
Indeed, the main task of \ocamlweb, after the pretty-printing job
consisting in making a readable documentation of the code, is to
generate a global index of cross-references, where each identifier is
associated to the lists of sections where it is defined or used.


\paragraph{Escapings: code inside documentation and vice versa.}
The first feature you will require is the ability to quote code inside
documentation and conversely. The latter, namely documentation inside
code, is simply obtained by normal \Caml\ comments filled in with a
\LaTeX\ contents. The former is obtained by quoting code between the
delimiters \texttt{[} and \texttt{]}. Square brackets may be nested,
the inner ones being understood as being part of the quoted code (thus
you can quote a list expression like $[1;2;3]$ by writing
\texttt{[[1;2;3]]}). Inside quotations, the code is pretty-printed in
the same way as it is in code parts.


\paragraph{Controls.}
In addition to the default behavior of \ocamlweb, you can control it
through a small set of commands, which are comments of a particular
shape. These commands are the following:
\begin{description}

\item[\texttt{(*s}] ~\par
  
  Starts a new section.

\item[\texttt{(*i} \quad\dots\quad \texttt{i*)}] ~\par
  
  Ignores all the text between those two delimiters.  Such ``comments''
  cannot be nested but any \Caml\ code and comments may appear between
  them, included nested \Caml\ comments.  You can use those delimiters
  to enclose a comment that shouldn't appear in the document, but also
  to hide some \Caml\ code, putting it between \texttt{(*i*)} and
  \texttt{(*i*)} in such a way that your code still compiles but is
  ignored by \ocamlweb. (Notice that of you use the delimiters
  \texttt{(*i} and \texttt{i*)} directly, the text inside is commented
  for both \Caml\ and \ocamlweb.)

\item[\texttt{(*c}] ~\par
  
  Tells \ocamlweb\ that this is a real \Caml\ comment an not a
  documentation text. Mainly useful is you want to insert comments
  after empty lines. However, this is not the spirit of \ocamlweb, and
  you are encouraged to put the documentation in separate \LaTeX\ 
  paragraphs and not in traditional \Caml\ comments.

\end{description}

%%%%%%%%%%%%%%%%%%%%%%%%%%%%%%%%%%%%%%%%%%%%%%%%%%%%%%%%%%%%

\section{Hints to get a pretty document}

\ocamlweb\ is rather robust, in the sense that it will always succeed
in producing a document, as soon as the \Caml\ files are lexically
correct. However, you may find the result rather ugly if so was your
source files. Here are some style hints to get a pretty document.

First, \ocamlweb\ is not a real code pretty-printer: it uses different
faces for identifiers and keywords, and use some mathematical symbols
for \verb!->!, \verb!*!, etc. but it does not indent your code. It
does not even cut your code lines when they are too long. The code
will always appear formatted as it is in the source files. The
indentation at the beggining of each line is output as a proportional
space (tabulations are correctly translated). In particular, if you
want your pattern-matchings to be aligned, you have to put a `\verb!|!'
also in front of the first pattern. Here is the difference:
\begin{displaymath}
  \begin{array}{l}
    \mathsf{let}~\mathit{f}~=~\mathsf{function} \\
    \kern2em\mathit{O}~\rightarrow~\dots \\
    \kern1em|~(\mathit{S}~p)~\rightarrow~\dots
  \end{array}
  \qquad\qquad
  \begin{array}{l}
    \mathsf{let}~\mathit{f}~=~\mathsf{function} \\
    \kern1em|~\mathit{O}~\rightarrow~\dots \\
    \kern1em|~(\mathit{S}~p)~\rightarrow~\dots
  \end{array}
\end{displaymath}


%%%%%%%%%%%%%%%%%%%%%%%%%%%%%%%%%%%%%%%%%%%%%%%%%%%%%%%%%%%%

\section{Usage}

\ocamlweb\ is invoked on a shell command line as follows:
\begin{displaymath}
  \texttt{ocamlweb }<\textit{options and files}>
\end{displaymath}
Any command line argument which is not an option is considered to be a
file (even if it starts with a \verb!-!). \Caml\ files are identified
by the suffixes \verb!.ml! and \verb!.mli!, and \LaTeX\ files by the
suffix \verb!.tex!. The latter will be copied `as is' in the final
document. The order of files on the command line is kept in the final
document.

\subsection*{Command line options}

\begin{description}

\item[\texttt{-o }\textit{file}, \texttt{--output }\textit{file}] ~\par
  
  Redirects the output into the file `\textit{file}'.

\item[\texttt{--no-web}] ~\par

  No \WEB\ style. In that case, there are no sections, and the user
  structurates his document directly through \LaTeX\ commands, like
  for instance \texttt{\symbol{92}section},
  \texttt{\symbol{92}subsection}, etc.
  There is still an index, in the usual \LaTeX\ sense. (Therefore, if
  you have introduced \LaTeX\ sections and subsections, their numbers
  will appear in the index.)

\item[\texttt{--no-index}] ~\par
  
  Do not print the index.

\item[\texttt{--extern-defs}] ~\par

  Keeps the external definitions in the index i.e. the identifiers
  which are not defined in any part of the code. (The default behavior
  is to suppress them from the index, even if they are used somewhere
  in the code.)

\item[\texttt{--header}] ~\par

  Does not skip the header of \Caml\ files. The default behavior is to
  skip them, since there are usually made of copyright and license
  informations, which you do not want to see in the final document.
  Headers are identified as comments right at the beginning of the
  \Caml\ file, and are stopped by any character other then a space
  outside a comment or by an empty line. 

\item[\texttt{--no-doc}] ~\par

  Suppresses the header and trailer of the final document. Thus, you can
  insert the resulting document into a larger one.

\item[\texttt{--latex-option }\textit{option}] ~\par

  Passes the given option the \LaTeX\ package \texttt{ocamlweb.sty}. 

\item[\texttt{--impl }\textit{file}, \texttt{--intf }\textit{file}, 
      \texttt{--tex }\textit{file}] ~\par

  Considers the file `\textit{file}' respectively as a \verb!.ml! file, a
  \verb!.mli! file or a \verb!.tex! file.

\item[\texttt{-h}, \texttt{--help}] ~\par

  Gives a short summary of the options and exit.

\item[\texttt{-v}, \texttt{--version}] ~\par

  Prints the version and exit.

\end{description}

%%%%%%%%%%%%%%%%%%%%%%%%%%%%%%%%%%%%%%%%%%%%%%%%%%%%%%%%%%%%

\section{FAQ}

\begin{enumerate}
\item \textbf{How can I insert `usepackage' commands, or whatever else,
  in the \LaTeX\ preamble?} ~\par

  Use the option \texttt{--no-doc} and catenate the result with your
  own header and trailer.

\item \textbf{How can I customize the appearance of the final
    document?} ~\par 

  Make your own version of the \LaTeX\ package
  \texttt{ocamlweb.sty}. There are macros for keywords, identifiers,
  index entries, etc. with short comments explaining their role.

\item \textbf{What about lexers?} ~\par

  \textsf{ocamllex} files are identified by the suffix \verb!.mll! and
  considered as implementation files. If \textit{file.mll} is such a
  file, then the cross reference engine will traverse the file
  \textit{file.ml}, if any. It is up to you to produce it using
  \textsf{ocamllex} before calling \ocamlweb. (Notice that the symbol
  \verb!*! in regular expressions appears as $\times$ in the final
  document.)

\item \textbf{What about parsers?} ~\par

  This is currently no support for \textsf{ocamlyacc} parsers.

\end{enumerate}



%%%%%%%%%%%%%%%%%%%%%%%%%%%%%%%%%%%%%%%%%%%%%%%%%%%%%%%%%%%%

\end{document}

%%% Local Variables: 
%%% mode: latex
%%% TeX-master: t
%%% End: 
