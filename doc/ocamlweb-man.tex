\documentclass[12pt]{article}

\usepackage[T1]{fontenc}
\usepackage[latin1]{inputenc}
\usepackage{fullpage}
\usepackage{url}

\newcommand{\WEB}{\textsf{WEB}}
\newcommand{\Caml}{\textsf{Caml}}
\newcommand{\OCaml}{\textsf{Objective Caml}}
\newcommand{\ocamlweb}{\textsf{ocamlweb}}
\newcommand{\monurl}[2]{\url{#1}}
%HEVEA\renewcommand{\monurl}[2]{\url{#2}{#2}}


\begin{document}

%%% titre %%%%%%%%%%%%%%%%%%%%%%%%%%%%%%%%%%%%%%%%%%%%%%%%%%
\title{ocamlweb: a literate programming tool \\ 
               for Objective Caml}
\author{Jean-Christophe Filli\^{a}tre and Claude March\'e \\
        \normalsize\monurl{http://www.lri.fr/~filliatr/ocamlweb}{http://www.lri.fr/\~{}filliatr/ocamlweb}}
\date{}
\maketitle

\tableofcontents

%%%%%%%%%%%%%%%%%%%%%%%%%%%%%%%%%%%%%%%%%%%%%%%%%%%%%%%%%%%

\section{Introduction}

Literate programming has been introduced by D.~E.~Knuth in 1984.  The
main idea is to put a code and its documentation in a single file and
to produce from it a document which is readable for a human, and not
only for a machine.
Although \ocamlweb\ borrows a lot of ideas from Knuth's original tool
(called \WEB), there are big differences between them. First, \WEB\ 
allows you to present the pieces of your code in any order, and this
is quite useful when using poorly structured languages, like
\textsf{Pascal} or \textsf{C}. But \OCaml\ is already highly
structured, and this is no more useful. Moreover, \WEB\ requires the
use of a tool to produce the code from the \WEB\ file, which greatly
complicates the use of your favorite source-based tools (dependencies
generator, debugger, emacs mode, etc.). When using \ocamlweb, the
documentation is inserted in the code as comments (in the \Caml\ 
sense), and your code is not linked to the existence of \ocamlweb\ in
any way.

Currently, the task of \ocamlweb\ may be seen as:
\begin{enumerate}
\item making a readable documentation of the code;
\item generating a global index of cross-references, where each
  identifier is associated to the lists of sections where it is
  defined or used.
\end{enumerate}


%%%%%%%%%%%%%%%%%%%%%%%%%%%%%%%%%%%%%%%%%%%%%%%%%%%%%%%%%%%%

\section{Principles}

Documentation is inserted into \Caml\ files as \emph{comments}.  Thus
your files may compile as usual, whether you use \ocamlweb\ or not.
\ocamlweb\ presupposes that the given \Caml\ files are well-formed (at
least lexically, but also syntactically if you want the cross
references to be correct).  \ocamlweb\ understands each \Caml\ file as
a sequence of paragraphs, each paragraph being either a piece of code or a
piece of documentation.  Documentation starts and ends with \Caml\ 
comment's delimiters \texttt{(*} and \texttt{*)}, and is a regular
\LaTeX\ text. Code starts with any characters sequence other than
\texttt{(*} and ends with an empty line.

Two styles are available for the final document:
\begin{itemize}
  
\item a \WEB\ style, where the code and documentation are displayed in
  \emph{sections}, numbered consecutively starting with~1.  A new
  section is started at the beginning of each file. You can also start
  a new section manually (see the paragraph below about controls);

\item a \LaTeX\ style, where you structure your document by yourself,
  using the usual sectioning commands of \LaTeX. 

\end{itemize}


\paragraph{Escapings: code inside documentation and vice versa.}
The first feature you will require is the ability to quote code inside
documentation and conversely. The latter, namely documentation inside
code, is simply obtained by normal \Caml\ comments filled in with a
\LaTeX\ contents. The former is obtained by quoting code between the
delimiters \texttt{[} and \texttt{]}. Square brackets may be nested,
the inner ones being understood as being part of the quoted code (thus
you can quote a list expression like $[1;2;3]$ by writing
\texttt{[[1;2;3]]}). Inside quotations, the code is pretty-printed in
the same way as it is in code parts.


\paragraph{Controls.}
In addition to the default behavior of \ocamlweb, you can control it
through a small set of commands, which are comments of a particular
shape. These commands are the following:
\begin{description}

\item[\texttt{(*s}] ~\par
  
  Starts a new section. (Meaningless in \LaTeX\ style.)

\item[\texttt{(*i} \quad\dots\quad \texttt{i*)}] ~\par
  
  Ignores all the text between those two delimiters.  Such ``comments''
  cannot be nested but any \Caml\ code and comments may appear between
  them, included nested \Caml\ comments.  You can use those delimiters
  to enclose a comment that shouldn't appear in the document, but also
  to hide some \Caml\ code, putting it between \texttt{(*i*)} and
  \texttt{(*i*)} in such a way that your code still compiles but is
  ignored by \ocamlweb. (Notice that of you use the delimiters
  \texttt{(*i} and \texttt{i*)} directly, the text inside is commented
  for both \Caml\ and \ocamlweb.)

\item[\texttt{(*c}] ~\par
  
  Tells \ocamlweb\ that this is a real \Caml\ comment an not a
  documentation text. Mainly useful is you want to insert comments
  after empty lines. However, this is not the spirit of \ocamlweb, and
  you are encouraged to put the documentation in separate \LaTeX\ 
  paragraphs and not in traditional \Caml\ comments.

\end{description}

%%%%%%%%%%%%%%%%%%%%%%%%%%%%%%%%%%%%%%%%%%%%%%%%%%%%%%%%%%%%

\section{Hints to get a pretty document}

\ocamlweb\ is rather robust, in the sense that it will always succeed
in producing a document, as soon as the \Caml\ files are lexically
correct. However, you may find the result rather ugly if so was your
source files. Here are some style hints to get a pretty document.

First, \ocamlweb\ is not a real code pretty-printer: it uses different
faces for identifiers and keywords, and use some mathematical symbols
for \verb!->!, \verb!*!, etc. but it does not indent your code. It
does not even cut your code lines when they are too long. The code
will always appear formatted as it is in the source files. The
indentation at the beggining of each line is output as a proportional
space (tabulations are correctly translated). In particular, if you
want your pattern-matchings to be aligned, you have to put a `\verb!|!'
also in front of the first pattern. Here is the difference:
\begin{displaymath}
  \begin{array}{l}
    \mathsf{let}~\mathit{f}~=~\mathsf{function} \\
    \hspace*{2em}\mathit{O}~\rightarrow~\dots \\
    \hspace*{1em}|~(\mathit{S}~p)~\rightarrow~\dots
  \end{array}
  \qquad\qquad
  \begin{array}{l}
    \mathsf{let}~\mathit{f}~=~\mathsf{function} \\
    \hspace*{1em}|~\mathit{O}~\rightarrow~\dots \\
    \hspace*{1em}|~(\mathit{S}~p)~\rightarrow~\dots
  \end{array}
\end{displaymath}


%%%%%%%%%%%%%%%%%%%%%%%%%%%%%%%%%%%%%%%%%%%%%%%%%%%%%%%%%%%%

\section{Usage}

\ocamlweb\ is invoked on a shell command line as follows:
\begin{displaymath}
  \texttt{ocamlweb }<\textit{options and files}>
\end{displaymath}
Any command line argument which is not an option is considered to be a
file (even if it starts with a \verb!-!). \Caml\ files are identified
by the suffixes \verb!.ml! and \verb!.mli!, and \LaTeX\ files by the
suffix \verb!.tex!. The latter will be copied `as is' in the final
document. The order of files on the command line is kept in the final
document.

\subsection*{Command line options}

%%% attention : -- dans un argument de \texttt est affich� comme un
%%% seul - d'o� l'utilisation de la macro suivante
\newcommand{\mm}{\symbol{45}\symbol{45}}

\begin{description}

\item[\texttt{-o }\textit{file}, \texttt{\mm{}output }\textit{file}] ~\par
  
  Redirects the output into the file `\textit{file}'.

\item[\texttt{\mm{}latex-sects}] ~\par
  
  In that case, there are no sections � la WEB, and the user
  structurates his document directly through \LaTeX\ commands, like
  for instance \verb|\section|, \verb|\subsection|, etc.  There is
  still an index, in the usual \LaTeX\ sense. (Therefore, if you have
  introduced \LaTeX\ sections and subsections, their numbers will
  appear in the index.)

\item[\texttt{\mm{}no-index}] ~\par
  
  Do not output the index.

\item[\texttt{\mm{}extern-defs}] ~\par

  Keeps the external definitions in the index i.e. the identifiers
  which are not defined in any part of the code. (The default behavior
  is to suppress them from the index, even if they are used somewhere
  in the code.)

\item[\texttt{\mm{}header}] ~\par

  Does not skip the header of \Caml\ files. The default behavior is to
  skip them, since there are usually made of copyright and license
  informations, which you do not want to see in the final document.
  Headers are identified as comments right at the beginning of the
  \Caml\ file, and are stopped by any character other then a space
  outside a comment or by an empty line. 

\item[\texttt{\mm{}no-preamble}] ~\par

  Suppresses the header and trailer of the final document. Thus, you can
  insert the resulting document into a larger one.

\item[\texttt{\mm{}latex-option }\textit{option}] ~\par

  Passes the given option the \LaTeX\ package \texttt{ocamlweb.sty}. 

\item[\texttt{\mm{}impl }\textit{file}, \texttt{\mm{}intf }\textit{file}, 
      \texttt{\mm{}tex }\textit{file}] ~\par

  Considers the file `\textit{file}' respectively as a \verb!.ml! file, a
  \verb!.mli! file or a \verb!.tex! file.

\item[\texttt{\mm{}files }\textit{file}] ~\par

  Read file names to process in file `\textit{file}' as if they were
  given on the command line. Useful for program sources splitted in
  several directories. See FAQ.
  
\item[\texttt{\mm{}no-greek}] ~\par
  
  Disable use of greek letters for single-letter type variables. For
  example, the declaration of \verb|List.length| is displayed as
  \begin{quote}
    \textit{List.length} : $\alpha$ \textit{list} $\rightarrow$
    $\alpha$
  \end{quote}
  but with this option, it will be displayed as
  \begin{quote}
    \textit{List.length} : \textit{'a} \textit{list} $\rightarrow$
    \textit{'a}
  \end{quote}

\item[\texttt{-h}, \texttt{\mm{}help}] ~\par

  Gives a short summary of the options and exit.

\item[\texttt{-v}, \texttt{\mm{}version}] ~\par

  Prints the version and exit.

\end{description}

%%%%%%%%%%%%%%%%%%%%%%%%%%%%%%%%%%%%%%%%%%%%%%%%%%%%%%%%%%%%

\section{The \texttt{ocamlweb} \LaTeX{} style file}
\label{section:ocamlweb.sty}

In case you choose to produce a document without the default \LaTeX{}
preamble (by using option \verb|--no-preamble|), then you must insert
into your own preamble the command
\begin{quote}
  \verb|\usepackage[|\textit{options}\verb|]{ocamlweb}|
\end{quote}
Alternatively, you may also pass these options with the
\verb|--latex-options| option of the \verb|ocamlweb| command.

The options that you may pass to the package are the following:

\begin{description}

\item[\texttt{latex-sects}] ~
  
  Tells \verb|ocamlweb.sty| that the document was generated with the
  \verb|--latex-sects| option so that no WEB sections are used, and
  the index was generated by referencing the LaTeX sections,
  subsections, etc.

\item[\texttt{novisiblespaces}] ~
  
  By default, spaces in strings of CAML code parts are output as
  \texttt{\char`\ }. They will be output as real spaces if you select
  this option.

\item[\texttt{bypages}] ~
  
  When sectioning is done by \LaTeX{} sectioning commands, this option
  asks that the index must refer to page numbers instead of section
  numbers. In WEB sectioning style, this option has no effect.

\end{description}

Additionally, you may alter the rendering of the document by
redefining some macros:
\begin{description}

\item[\texttt{ocwkw}, \texttt{ocwbt}, \texttt{ocwupperid},
  \texttt{ocwlowerid}, \texttt{ocwtv}] ~ 
  
  The one-argument macros for typesetting keywords, base types (such
  as \verb|int|, \verb|string|, etc.), uppercase identifiers (type
  constructors, exception names, module names), lowercase identifiers
  and type variables respectively. Defaults are sans-serif for
  keywords and italic for all others. Some of the single-letter type
  variables are displayed by default as greek letters, but this
  behaviour can be selected by the \verb|--no-greek| option.

  For example, if you would like a slanted font for base types, you
  may insert  
\begin{verbatim}
     \renewcommand{\ocwbt}[1]{\textsl{#1}}
\end{verbatim}
  anywhere between \verb|\usepackage{ocamlweb}| and
  \verb|\begin{document}|. 

\item[\texttt{ocwinterface}, \texttt{ocwmodule},
  \texttt{ocwinterfacepart}, \texttt{ocwcodepart}] ~ 
  
  One-argument macros for typesetting the title of a \verb|.mli| file,
  the title of a \verb|.ml| file, the interface part of a \verb|.ml|
  file and the code part of a \verb|.ml| file, respectively. Defaults
  are is
\begin{verbatim}
\newcommand{\ocwinterface}[1]{\section*{Interface for module #1}}
\newcommand{\ocwmodule}[1]{\section*{Module #1}}
\newcommand{\ocwinterfacepart}{\subsection*{Interface}}
\newcommand{\ocwcodepart}{\subsection*{Code}}
\end{verbatim}
  and you may redefine them using \verb|\renewcommand|.

\end{description}

%%%%%%%%%%%%%%%%%%%%%%%%%%%%%%%%%%%%%%%%%%%%%%%%%%%%%%%%%%%%

\section{FAQ}

\begin{enumerate}

\item \textbf{What about an HTML output?} ~\par

  Use \textsf{hevea}, the \LaTeX\ to HTML translator written by Luc
  Maranget (freely available at
  \monurl{http://pauillac.inria.fr/hevea/}{http://pauillac.inria.fr/hevea/}), 
  with the following command-line:
  \begin{displaymath}
    \texttt{hevea} ~ \texttt{ocamlweb.sty} ~ \textit{file.tex}
  \end{displaymath}
  where \textit{file.tex} is the document produced by \ocamlweb.
  The package \texttt{ocamlweb.sty} contains the necessary support for
  \textsf{hevea}.

\item \textbf{How can I customize the appearance of the final
    document?} ~\par 

  %Make your own version of the \LaTeX\ package
  %\texttt{ocamlweb.sty}. There are macros for keywords, identifiers,
  %index entries, etc. with short comments explaining their role.
  % jcf, tu nous fais le coup a chaque fois: tu sais bien que ce n'est
  % pas la bonne mani�re de faire car l'utilisateur ne peut plus se
  % mettre a jour !
  
  You can redefine some of the \LaTeX{} macros of the
  \verb|ocamlweb.sty| style file. See
  Section~\ref{section:ocamlweb.sty}. We do not recommend to modify
  the \verb|ocamlweb.sty| file directly, since you would not be able to
  update easily to future versions. You should rather redefine the
  macros using \verb|\renewcommand|. If you want to customize other
  parameters that are not currently customizable, please contact the
  developers.
    
\item \textbf{How can I insert `usepackage' commands, or whatever else,
  in the \LaTeX\ preamble?} ~\par

  Use the option \texttt{\mm{}no-preamble} and catenate the result with your
  own header and trailer. (Or use a \verb|\input| or \verb|\include|
  command to insert the file generated by \ocamlweb\ in your main
  LaTeX file.)

\item \textbf{What about lexers and parsers?} ~\par

  This is currently no real support for \textsf{ocamllex} lexers and
  \textsf{ocamlyacc} parsers. You can pass the corresponding files to
  \ocamlweb\ through the option \verb!--impl!, and the produced
  \LaTeX\ file should compile. However, there will be no indexing for
  the identifiers of those modules and the pretty-print may look
  somewhat dirty (The symbol \verb!*! will always appear as $\times$,
  in the final document, even when it appears in regular expressions
  in lexers or in the comments \verb!/*! and \verb!*/! in parsers.)

\item \textbf{What can I do with a large program with sources in several
    directories, organised as libraries?} ~

  First alternative: you can run an ocamlweb command in each
  directories, producing a documentation for each library. If you want
  to have the documentations alltogether in the same TeX file, run
  each ocamlweb commands with option \texttt{\mm{}no-preamble} to avoid
  generation of the LaTeX preamble, build by hand a master document
  file with the preamble you want, without forgetting the
  \verb|\usepackage{ocamlweb}|, and input each file in each directory
  with the \verb|\input| or the \verb|\include| macro.
  
  Second alternative: the main problem with the first alternative is
  that you will have an separate index for each libraries. If you want
  a global index, you need the run only one \verb|ocamlweb| command.
  If you don't want to put all source file names on one single command
  line, you can use the \verb|--files| option which allows you to read
  the source file names from a file. Example :
\begin{verbatim}
    ocamlweb --files lib1/source-files --files lib2/source-files
\end{verbatim}
  will produce documentation for files in \verb|lib1| and \verb|lib2|
  provided that the files \verb|source-files| in each directory
  contain the name of the documented source files.

\end{enumerate}



%%%%%%%%%%%%%%%%%%%%%%%%%%%%%%%%%%%%%%%%%%%%%%%%%%%%%%%%%%%%

\end{document}

%%% Local Variables: 
%%% mode: latex
%%% TeX-master: t
%%% End: 
